\documentclass{article}

\usepackage[margin=0.5in]{geometry}
\usepackage{amsmath, amssymb, changepage, amsthm}

\begin{document}

\newcommand{\N}{\mathbb{N}}
\newcommand{\Z}{\mathbb{Z}}
\newcommand{\Q}{\mathbb{Q}}
\newcommand{\R}{\mathbb{R}}
\newcommand{\T}{\text{\normalfont\ T}}
\newcommand{\F}{\text{\normalfont\ F}}
\newcommand{\ti}{\textit}
\newcommand{\tb}{\textbf}
\newcommand{\n}{\leavevmode \newline}
\newcommand{\nn}{\leavevmode \newline \newline}
\def \Def#1#2{\begin{adjustwidth}{0.85cm}{0.85cm} \tb{(Definition) #1}: \ti{#2} \end{adjustwidth}}
\def \nDef#1#2{\n \Def{#1}{#2}}
\def \Defn#1#2{\Def{#1}{#2} \n}
\def \nDefn#1#2{\n \Defn{#1}{#2}}
\def \Defcont#1{\begin{adjustwidth}{0.85cm}{0.85cm} \ti{#1} \end{adjustwidth} \n}
\def \InDef#1{\ti{\begin{adjustwidth}{0.85cm}{0.85cm} #1 \end{adjustwidth}}}
\def \Thm#1#2{\begin{adjustwidth}{0.85cm}{0.85cm} \tb{(Theorem) #1}: \ti{#2} \end{adjustwidth}}
\def \nThm#1#2{\n \Thm{#1}{#2}}
\def \Thmn#1#2{\Thm{#1}{#2} \n}
\def \nThmn#1#2{\n \Thmn{#1}{#2}}
\def \InThm#1{\ti{\begin{adjustwidth}{0.85cm}{0.85cm} #1 \end{adjustwidth}}}
\def \Pf#1{\begin{adjustwidth}{0.85cm}{0.85cm} \textit{Proof}: #1 \qedsymbol \end{adjustwidth} \n}
\newcommand{\st}{\textnormal{ s.t. }}
\newcommand{\proplang}{\mathcal{L}_0}
\newcommand{\predlang}{\mathcal{L}}
\newcommand{\M}{\mathcal{M}}
\newcommand{\A}{\mathcal{A}}
\newcommand{\LA}{\predlang_\A}

\title{Logic of First Order Structures}
\author{Piyush Patil}
\date{October 9, 2017}
\maketitle

\section{Isomorphisms between structures}
We begin by defining isomorphisms, i.e. structure-preserving bijective mappings, between structures.
\nDef{$ \LA $-structure isomorphism}{Let $ \M = (M, I) $ and $ \mathcal{N} = (N, J) $ be $ \LA $-structures. An \tb{isomorphism} between $ \M $ and $ \mathcal{N} $ is a bijective mapping $ f: M \rightarrow N $ for which the following hold:}
\InDef{\begin{enumerate}
    \item Preservation of constant symbols: For each constant $ c_i $ in the domain of $ I $,
        $$ f(I(c_i)) = J(c_i) $$
    \item Preservation of function symbols: For each function symbol $ F_i $ in the domain of $ I $,
        $$ \ti{ for every } a_1, \cdots, a_{\pi(F_i)}, b \in M: I(F_i)(a_1, \cdots, a_{\pi(F_i)}) = b \ti{ if and only if } J(F_i)(f(a_1), \cdots, f(a_{\pi(F_i)})) = f(b) $$
    \item Preservation of predicate symbols: For each preservation symbol $ P_i $ in the domain of $ I $,
        $$ \ti{ for every } a_1, \cdots, a_{\pi(P_i)} \in M: (a_1, \cdots, a_{\pi(P_i)}) \in I(P_i) \ti{ if and only if } (f(a_1), \cdots, f(a_{\pi(P_i)})) \in J(P_i) $$
\end{enumerate}}
We often write $ \M \cong \mathcal{N} $ to denote that $ \M $ is isomorphic to $ \mathcal{N} $. As usual, a morphism is simply a structure-preserving mapping between general mathematical objects (or more precisely, categories), and an isomorphism is a sufficiently strong such mapping for which we can conclude that the two objects are in fact the same, structurally. Given two structures for which we can associate every object in the universe of one with a unique object in the universe of the other, and moreover for which the interpretation of constant, function, and predicate symbols is the same with respect to the respective objects, it follows that the structures are merely relabelings of each other. Note that isomorphisms are closed under composition and inverses.
\nn
To emphasize that isomorphisms are structure-preserving, it should be true that they preserve satisfiability. We prove this in the following theorem, which tells us that if we replace a universe's objects with the associated (through an isomorphism) objects of a different universes, satisfiability is preserved under the image of the isomorphism. In general, remapping objects like this might destroy the structure and render certain formulas satisfied or unsatisfied differently from the source universe, but because isomorphisms are designed to preserve logical structure, this is not the case.
\nThm{Isomorphisms preserve satisfiability}{Suppose $ f: M \rightarrow N $ is an isomorphism between the $ \LA $-structures $ \M = (M, I) $ and $ \mathcal{N} = (N, J) $. If $ \nu $ is any $ \M $-assignment, then the composition $ f \circ \nu $ is an $ \mathcal{N} $-assignment, and further}
$$ \ti{ for every $ \LA $-formula $ \phi $: } (\M, \nu) \vDash \phi \ti{ if and only if } (\mathcal{N}, f \circ \nu) \vDash \phi $$
\Pf{It's clear that $ f \circ \nu $ is an $ \mathcal{N} $-assignment, since it first uses $ \nu $ to map variables to objects in $ M $, which are then uniquely mapped to objects in $ N $ by $ f $. Our outline for proving that the composition preserves satisfiability is to first prove, by induction, that the composition of an assignment with an isomorphism preserves the interpretation of terms, and then use that result to prove the theorem itself, by induction.
\n
Recall that given our assignment $ \nu $, which assigns variables to objects in $ M $, the extended assignment leverages our variable assignments to assign more general terms to objects in $ M $. We wish to prove that $ f \circ \overline{\nu} = \overline{f \circ \nu} $. Let's proceed by induction on the length of a term. The base case, then, is analogous to the case of atomic terms, in which case
    $$ (f \circ \overline{\nu})(\langle c_i \rangle) = f(\overline{\nu}(\langle c_i \rangle)) = f(I(c_i)) = J(c_i) = \overline{f \circ \nu}(\langle c_i \rangle) $$
since all extended assignments over $ N $ by definiton map constants to $ J(c_i) $, or
    $$ (f \circ \overline{\nu})(\langle x_i \rangle) = f(\overline{\nu}(\langle x_i \rangle)) = f(\nu(x_i)) = (f \circ \nu)(x_i) = \overline{f \circ \nu}(x_i) $$
Next, suppose that indeed the interpretation of terms if preserved for terms of length less than that of a fixed term $ \tau $. Since $ \tau $ is not atomic by assumption, we can write $ \tau = F_i(\tau_1, \cdots, \tau_n) $ for terms $ \tau_j, 1 \leq j \leq n $, shorter than $ \tau $. Then
    $$ \begin{aligned}
        (f \circ \overline{\nu})(\tau) &= f(\overline{\nu}(F_i(\tau_1, \cdots, \tau_n))) = f(I(F_i)(\overline{\nu}(\tau_1), \cdots, \overline{\nu}(\tau_n))) = f(J(F_i)(f(\overline{\nu}(\tau_1)), \cdots, f(\overline{\nu}(\tau_n)))) \\
        &= f(J(F_i)(\overline{f \circ \nu}(\tau_1), \cdots, \overline{f \circ \nu}(\tau_n))) = \overline{f \circ \nu}(F_i(\tau_1, \cdots, \tau_n))
    \end{aligned} $$
This proves that $ \overline{f \circ \nu} = f \circ \overline{\nu} $.
\n
Next, let's use this result to prove the analogous result on formulas, again on induction on the length of formulas. Again, the base case corresponds to atomic formulas, which easily follow: TODO}
The above theorem suggests that isomorphic structures are equivalent, at least in terms of which formulas they satisfy and which they don't. More trivially, if it's the case that two structures satisfy precisely the exact same sentences (a special case of the above theorem that's independent of assignments), we say the structures, say $ \M $ and $ \mathcal{M} $, are \ti{elementarily equivalent}, and write $ \M \equiv \mathcal{N} $.

\section{Substructures and elementary substructures}
In this section, we consider structures that contain other structures, and characterize different notions of containment and ways of determining substructure relationships.
\nDef{Substructure}{Let $ \M = (M, I) $ and $ \mathcal{N} = (N, J) $ be $ \LA $-structures. $ \M $ is a \tb{substructure} of $ \mathcal{N} $, denoted $ \M \subseteq \mathcal{N} $, if $ M \subseteq N $ and the following hold:}
\InDef{\begin{enumerate}
    \item Interpretations agree on constants: $ \forall c_i \in \mathcal{A}: I(c_i) = J(c_i) $.
    \item Interpretations agree on functions: $ \forall F_i \in \mathcal{A}: I(F_i) \text{ is the restriction of } J(F_i) \text{ to } M $.
    \item Interpretations agree on predicates: $ \forall P_i \in \mathcal{A}: I(P_i) = J(P_i) \cap M^{\pi(P_i)} $.
\end{enumerate}}
\n
Thus, a substructure is a substructure for which its containing superstructure restricts down to the structure completely.
\nn
It turns out that substructure relationships hold precisely when we have a form of elementary equivalence, which is strong enough to automatically guarantee the above properties.
\nThm{Substructures prove the same atomic formulas}{Let $ \M = (M, I) $ and $ \mathcal{N} = (N, J) $ be $ \LA $-structures with $ M \subseteq N $. Then $ \M \subseteq \mathcal{N} $ if and only if}
\InThm{
    $$ \forall \text{ atomic $ \LA $-formulas $ \phi $, $ \M $-assignments $ \nu $}: (\M, \nu) \vDash \phi \iff (\mathcal{N}, \nu) \vDash \phi $$
}
\Pf{TODO}
This theorem inspires the following definition, which formalizes the above property as a similar but distinct property from elementary equivalence.
\nDef{Elementary substructure}{Let $ \M $ and $ \mathcal{N} $ be $ \LA $-structures with $ \M \subseteq \mathcal{N} $. $ \M $ is an \tb{elementary substructure} of $ \mathcal{N} $ if}
\InDef{
    $$ \forall \ti{$ \LA $-formulas $ \phi $, $ \M $-assignments $ \nu $}: (\M, \nu) \vDash \phi \iff (\mathcal{N}, \nu) \vDash \phi $$
In which case we write $ \M \preceq \mathcal{N} $.}

\section{Definable Sets and Tarski's Criterion}
Let's start this section by introducing definable sets. We'll give the formal definition below, followed by an explanation of the motivation.
\nDef{Definable set}{Let $ \M = (M, I) $ be an $ \LA $-structure. We say that a set $ Y \subseteq M^n $ is \tb{definable in $ \M $ with parameters from $ X $} if there exists a set $ X \subseteq M $ with elements $ b_1, \cdots, b_m \in X $, and an $ \LA $-formula $ \phi $, such that}
\InDef{\begin{enumerate}
    \item $ \phi $ has $ n + m $ free variables $ x_1, \cdots, x_{n + m} $
    \item For every $ (a_1, \cdots, a_n) \in M^n $, $ (a_1, \cdots, a_n) \in Y $ if and only if there's some $ \M $-assignment $ \nu $ such that
        $$ \nu(x_i) = \begin{cases}
            a_i, &\ti{ if } 1 \leq i \leq n \\
            b_i, &\ti{ if } n + 1 \leq i \leq m
        \end{cases} $$
    and $ (\M, \nu) \vDash \phi $.
\end{enumerate}
Further, we say $ Y $ is \tb{definable in $ \M $ without parameters} if $ X = \emptyset $.}
\n
Let's break this rather unwieldy definition down. Our motivation for what a definable set actually is comes from the following approach to naive set theory: informally, we conceptualize sets as collections of objects. Sets can be represented by explicitly enumerating their objects, but this approach has set theoretic shortcomings when we try to consider infinite, especially uncountable or larger, sets. Instead, sets are more often defined in terms of some kind of construction based on some property. Most generally, a set is defined as the collection of objects within some universe which satisfy some property. Symbolically, we may express this idea as
    $$ \ti{ for any set $ S $}: S = \{ x \st P(x) \} $$
for some predicate relation or proposition $ P $. Hence, set theory and logical systems are intimately related, as we require the former to lay the scaffolding of a truth system which allows us to formally construct predicates and propositions that we can use to define sets through some kind of filtering process over elements from some larger universe of elements (which itself is usually constructed axiomatically or taken as an undefined primitive). It may be initially unclear how this definition links to the motivation we just presented, and in particular how the component of definability with parameters fits in. The case of definability without parameters is simpler and easier to motivate, so we'll start with that before moving on to showing how the introduction of parameters is merely a generalization technique that enhances the expressivity of our logical system to define sets which are normally unable to be defined in the structure.
\nn
Recall that we initially motivated definability for sets from a set theoretic approach, in trying to answer the question of how to formally construct arbitrary, often infinite sets. To do so, we require some framework for a logical system that allows us to formally construct relations or some kind of logical filter for specifying which elements fit within our set and which do not. Within the context of first-order logic, $ \LA $-structures can be used in this regard. If we have some $ \LA $-structure $ \M = (M, I) $, which we wish to use to define sets, then the total universe of objects from which our sets are constructed is $ M $, with $ I $ specifying the semantics of the underlying logical system. So, our question now becomes which subsets of $ M $ are definable and which aren't. Since we want to define sets as the collection of objects from the universe $ M $ which together share some property, we'll use predicates for this purpose. In fact, we can identify predicates $ P_i \in \A $ with sets in and of themselves, since sets are wholly defined by the membership operation, which we can leverage a predicate symbol for by taking it's truth value as an indication of the membership criterion. Hence, we want, informally, sets $ S \subseteq M $ to be defined using some notion similar to
    $$ S = \{ m \in M \st m \in I(P_i(m)) \ti{ for some $ i $ } \} $$
Of course, it seems unnecessarily restrictive to be forced to define sets using only predicates; there's no reason why we can't chain predicates or other logical symbols together to form more complex propositions that serve as the filtering mechanism in defining $ S $. For this reason, the definition we really want is something of the form
    $$ S = \{ m \in M \st \M \vDash \phi[m] \ti{ for some $ \LA$-formula $ \phi(x_1) $ } \} $$
So, $ S $ is definable if we can fully characterize its set-theoretic structure, i.e. its membership criterion, by means of some logical formula. In other words, we define $ S $ as the objects in $ M $ which make $ \phi $ true, under the logical structure defined by $ \M $. Notice that the definition that $ S \subseteq M $ is definable if there exists a mapping $ \nu $ and $ \LA $-formula $ \phi $ with one free variable $ x_1 $ such that $ m \in M $ is in $ S $, where
    $$ \nu(x_1) = m $$
if and only if $ (\M, \nu) \vDash \phi $ is equivalent to the formal definition we began this section with, in the case $ n = 1 $ and $ X = \emptyset $. We easily note we can leverage this definition to a more powerful definition that can capture definability for even more sets, not just the subsets of $ M $, if we consider subsets of $ M^n $. Extending the definition from sets of elements of $ M $ to sets of $ n $-tuples from $ M $ is straightforward; we just allow $ \phi $ to have $ n $ free variables and force $ \nu $ to assign each to the corresponding object in the tuple we're considering. This amounts to the definition
    $$ S \subseteq M^n \ti{ is definable if } \exists \phi(x_1, \cdots, x_n) \in \LA \st S = \{ (m_1, \cdots, m_n) \in M^n \st \M \vDash \phi[m_1, \cdots, m_n] \} $$
One implication of this form of defining definability is that the same set might be definable under certain structures and languages, but not under others; the notion of definability depends critically on the underlying structure we use. More specifically, whether or not a set is definable depends on the expressivity of the underlying structure. A natural question to ask at this point is: if a set is not definable under some structure, is there some way we can extend the structure to a more expressive one under which our set can be defined? This leads us to the idea of the expansion of a structure. If we have some structure $ \M $ and some subset $ A $ of $ M $, we can augment the langauge $ \A $ as
    $$ \LA := \LA \cup \{ c_a, a \in A \} $$
    $$ I(c_a) := a $$
In other words, for each object in $ A $, we explicitly add a constant symbol to our language which we associate with the corresponding object in the augmented $ M $. This new structure, which we denote $ \M_a $, is called the \ti{expansion} of $ \M $, and our reasoning for doing this is to enhance the expressivity of our logical structure. In particular, we can now define every finite subset of $ A $ by virtue of the new constant symbols we added to our language. Further, because we only ever use finitely many symbols when defining a set, any definable set in $ \M_a $ is also definable in $ \M_{A_0} $ for some subset $ A_0 \subseteq A $. This is exactly what we mean by definability with parameters - even if a set $ S \subseteq M $ is not definable in the raw $ \M $, if we can expand $ \M $ to some $ \M_A $ then $ S $ is definable in $ \M $ if we only allow it to use parameters from $ A $, given by the finite $ A_0 \subseteq A $. Thus the "parameters" in use are simply new constant symbols we add to our language which allow us to construct a previously impossible formula that uniquely decides which elements are in $ S $. That is, even though we can't find any $ \phi(x_1) $ for which $ m \in S \iff \phi[m] $ is true in $ \M $, we might be able to generalize $ \phi $ to $ \phi(x_1, x_2, \cdots, x_{m + 1}) $ and subsequently define $ S $ using $ m \in S \iff \phi[m, a_1, \cdots, a_m] $ is true $ \M_{A_0} $ where $ A_0 = \{ a_1, \cdots, a_n \} $.
\nn
We'll often make use of the equivalent, but more elegant definition of definability that $ Y \subseteq M^n $ is definable in $ \M = (M, I) $ with parameters from $ X \subseteq M $ if
    $$ \exists b_1, \cdots, b_m \in X, \phi(x_1, \cdots, x_{n + m}) \in \LA \st Y = \big \{ (a_1, \cdots, a_n) \in M^n \st \M \vDash \phi[a_1, \cdots, a_n, b_1, \cdots, b_m] \big \} $$
If a set is defined with parameters, then the set's definition depends only on and is determined completely by those parameters. Thus, it follows that any rearrangement of the objects of our universe should preserve both the definability of a set and the actual members of the set, so long as the rearrangement (1) preserves the logical structure of the underlying universe and interpretation, and (2) leaves the parameters defining the set unchanged. These two conditions are sufficient to determine the preservation of a definable set across any rearrangement of the universe, since the definability of the set, and therefore the set itself, hinges only on those parameters and therefore leaving the parameters unchanged necessitates leaving the set invariant through the rearrangement. Formally, we have the following theorem.
\nThm{Automorphisms that preserve parameters preserve definable sets}{Let $ \M = (M, I) $ be an $ \LA $-structure. If $ Y \subseteq M^n $ is definable in $ \M $ with parameters from $ X \subseteq M $, then for any automorphism $ f $ on $ \M $ which is invariant on $ X $, $ Y $ is preserved under $ f $. That is,}
    $$ \ti{ $ \forall $ automorphisms $ f: M \rightarrow M $}: \forall b \in X: f(b) = b \rightarrow  Y = f(Y) := \{ (f(a_1), \cdots, f(a_n)) \ti{ for } (a_1, \cdots, a_n) \in Y \} $$
\Pf{By definition, we can find $ b_1, \cdots, b_m \in X $ and $ \phi(x_1, \cdots, x_{n + m}) $ such that
    $$ (a_1, \cdots, a_n) \in Y \iff \M \vDash \phi(a_1, \cdots, a_n, b_1, \cdots, b_m) $$
Let's first prove that $ Y \subseteq f(Y) $. For any $ (a_1, \cdots, a_n) \in Y $, since $ \phi(a_1, \cdots, a_n, b_1, \cdots, b_m) $ is true in $ \M $, we can unpack the definition of satisfiability so that for any $ \M $-assignment $ \nu $ for which $ \nu(x_i) = a_i, 1 \leq i \leq n $ and $ \nu(x_j) = b_j, 1 \leq j \leq m $, $ (\M, \nu) \vDash \phi $; since isomorphisms preserve satisfiability, in the sense that for any isomorphism $ f $ to structure $ \mathcal{N} $, $ (\mathcal{N}, f \circ \nu) \vDash \phi $. Thus, if $ f $ is an automorphism then it follows that
    $$ \M \vDash \phi[f(a_1), \cdots, f(a_n), f(b_1), \cdots, f(b_m)] = \phi[f(a_1), \cdots, f(a_n), b_1, \cdots, b_m] $$
where the last equality comes from $ X $ being invariant under $ f $. Thus, $ (f(a_1), \cdots, f(a_n)) \in Y $, and since every element of $ f(Y) $ can be expressed in this form, it follows that $ f(Y) \subseteq Y $.
\nn
Let's now prove that $ Y \subseteq f(Y) $. Given some $ (a_1, \cdots, a_n) \in Y $, we want to show that $ a_i = f(a_i') $ for some $ (a_1', \cdots, a_n') \in Y $. Since $ f $ is an automorphism, so is the inverse mapping $ f^{-1} $. This means that since $ \M \vDash \phi[a_1, \cdots, a_n, b_1, \cdots, b_m] $, we also have that $ \M \vDash \phi[f^{-1}(a_1), \cdots, f^{-1}(a_n), b_1, \cdots, b_m] $ by the same logic as before. Hence, $ (f^{-1}(a_1), \cdots, f^{-1}(a_n)) \in Y $, and so we can take $ a_i' := f^{-1}(a_i) $ and conclude that $ Y \subseteq f(Y) $. It follows that $ Y = f(Y) $ as desired.
}
Though definable sets are a cornerstone topic in logic in their own right, we'll conclude this section by studying one application of them; we'll use definable sets to prove a kind of litmus test for elementary substructures. Typically, there are reasonable criteria for determining if there's a substructure relationship between two structures, and even if the relationship is elementary or not. Indeed, we need only verify that the two structures satisfy precisely the same formulas. However, the problem of actually constructing an elementary substructure of a given structure is more difficult. The difficulty here is in dealing with formulas involving quantifiers, because quantifiers act not just on the subformula used to create it, but also on the entire universe of the structure. Thus, in attempting to construct a substructure which satisfies the same formulas as the enclosing structure, we're required to consider formulas whose satisfiability depends on the entire universe of our substructure, before we've even finished constructing the entire universe. This level of foresight and anticipation of the nature of our substructure as we construct it complicates the process. Fortunately, we can use definable sets to characterize elementary substructures.
\nThm{Tarski's criterion}{Let $ \M = (M, I) $ be a substructure of $ \mathcal{N} = (N, J) $. Then $ \M $ is elementary substructure of $ \mathcal{N} $ if and only if every (non-empty) subset of $ N $ that's definable with parameters from $ M $ contains elements from $ M $. That is,}
\InThm{
    $$ \M \preceq \mathcal{N} \iff \forall A \subseteq N: \ti{ if $ A $ is definable with parameters from $ M $, } A \cap M \neq \emptyset $$
}
\Pf{TODO}
Here's the intuition behind the above theorem, which is often called the \ti{Tarski-Vaught test}. The elementary substructure relationship requires both structures to satisfy the same formulas under any $ \M $-assignment. We can reframe this condition by recalling that every formula can be identified uniquely with a definable set - namely, consider the set whose membership criterion is decided by a formula. Hence, formulas and definable sets are really two side of the same coin, and we can alternate back and forth between these two perspectives. Clearly, if a substructure satisfies a formula then a superstructure must also satisfy it, since we defined the notion of substructures merely by restricting a superstructure down to a smaller, contained structure. For the elementary substructure condition to break, we need a true formula in the superstructure that's false in the substructure. Shifting to the perspective from definable sets, a formula being true in a structure simply means that its corresponding definable set is not empty. Of course, since we're considering truth and satisfiability only under $ \M $-assignments (looking at assignments that map onto elements not in $ M $ is meaningless in this context), it only makes sense to look at formulas corresponding to definable sets with no parameters or parameters from $ M $; if a definable set required parameters not from $ M $, then it would correspond to a formula that couldn't exist in $ \M $ and would instead require more machinery from the superstructure that doesn't exist in the substructure. Thus, if every definable set over $ \M $ (that is, definable with parameters only from $ M $) is non-empty in both $ \mathcal{N} $ and in $ \M $, then because these sets correspond (bijectively) to formulas that are true in both $ \mathcal{N} $ and $ \M $, it follows that $ \M $ and $ \mathcal{N} $ satisfy the same formulas.

\section{Dense Orders}
It's not too difficult to show that a set is definable, as we can just look for a formula that the set's elements commonly satisfy, usually by analyzing the structure of the set. However, this can become complex for larger structures, and it can be highly non-trivial to show that a set is not definable. In this section, we'll study, by means of an example, how to utilize the lemma that definable sets are invariant under automorphisms (that fix the set's defining parameters) to construct automorphisms on the fly which aid us in determining if a given set is definable.
\nn
Let's narrow our attention to structures over the language $ \LA = \{ P \} $ where $ P $ is a 2-ary predicate symbol. Then, all the $ \LA $-structures over some domain $ M $ are completely determined by the relation $ P $, so we denote the $ \LA $-structures as $ (M, P) $. The prototypical example here we'll use is $ (\R, <) $, the universe of real numbers under their usual ordering ($ a < b \iff a - b \in \R^+ $). Further, for any finite set $ X \subset \R $, consider the equivalence relation $ \sim_X $ given by
    $$ \ti{ for $ a, b \in \R $: } a \sim_X b \iff \exists \ti{ automorphism } e: (\R, <) \rightarrow (\R, <) \st e(a) = b \ti{ and } \forall t \in X: e(t) = t $$
That is, $ a $ is equivalent to $ b $ under this relation if we can move $ a $ to $ b $ in an order-preserving way that leaves $ X $ invariant. It's easy to see that this is an equivalence relation: (1) clearly, $ \forall x \in \R: x \sim_X x $ since the identity map is an automorphism that preserves $ X $, (2) $ x \sim_X y $ if and only if $ y \sim_X x $ since the inverse of an automorphism is an automorphism, and has the same invariant sets, and (3) $ x \sim_X y $ and $ y \sim_X z $ give us $ x \sim_X z $ since the composition of two automorphisms is an automorphism, which leaves $ X $ invariant since $ X $ is unchanged under both composing automorphisms. We denote the equivalence class of $ x $ under $ \sim_X $ with $ [x]_X $.
\nn
Intuitively, the only automorphisms that map $ a $ to $ b $ in an order-preserving way while leaving $ X $ untouched are maps that somehow slide the whole line around $ a $ over to $ b $. Formally, we'll use the familiar notion of an \ti{interval} to state this.
\nDef{Interval}{An \tb{interval} of $ \R $ is a set $ I $ for which the following holds:}
    $$ \forall a, b, c \in \R \st a \leq c \leq b: a \in I \ti{ and } b \in I \rightarrow c \in I $$
\InDef{The \tb{endpoints} of $ I $ are its supremum and infinimum. We use the standard notation}
    $$ (a, b) = \{ x \in \R \st a < x < b \} \ti{ and } [a, b] = \{ x \in \R \st a \leq x \leq b \} $$
This is the usual definition of an interval, as a continuous line segment in the reals. We can now show that all equivalence classes of $ \sim_X $ are intervals. Intuitively, we can arrive at this result by visualizing the underlying condition for the equivalence relation - an order-preserving mapping that fixes the (finitely many) points of $ X $. Since the mapping is order-preserving, we can visualize it as stretching and compressing the real line, but never twisting or crossing it in a way that would send a point after any of its successors. If we want points of $ X $ to be fixed, then we split the real line into regions that are separated by points from $ X $; in any such region between two points $ t_1, t_2 \in X $ (and contianing no points from $ X $), our mapping can only send a point to any of the points ahead of it, up to $ t_2 $ (assume that $ t_1 < t_2 $) or points below it, up to $ t_1 $. Hence, any point in that region is only equivalent to points within the interval $ (t_1, t_2) $. This is the content of the following theorem.
\nThm{Equivalence classes of $ \sim_X $ are intervals}{Let $ X \subseteq \R $ be finite and $ a \in \R $. If $ a \in X $, then $ [a]_X = \{ a \} $. Otherwise, $ [a]_X $ is the maximum interval in $ \R $ that contains $ a $ and contains no points of $ X $. That is,}
    $$ [a]_X = \ti{maximal interval $ I \subseteq \R $} \st a \in I \ti{ and } I \cap X = \emptyset $$
\Pf{If $ a \in X $ then the only order-preserving mappings that fix $ X $ must also fix $ a $, so $ a \sim_X b \rightarrow a = b $. Otherwise, if we define $ I $ as above, we can prove $ I = [a]_X $, for any $ b \in I $ we can take $ (c, d) \subseteq I $ such that $ a, b \in (c, d) $. Assuming $ b > a $ (otherwise, apply the following argument but with the following automorphism's inverse), define
    $$ e: \R \rightarrow \R \st \forall x \in \R: e(x) =
    \begin{cases}
        c + \frac{b - c}{a - c} (x - c), &\ti{ if } x \in [c, a] \\
        b + \frac{d - b}{d - a} (x - a), &\ti{ if } x \in [a, d] \\
        x, &\ti{ otherwise }
    \end{cases} $$
Then, $ e $ is order-preserving, and a bijection, and has $ e(a) = b $. Thus, $ a \sim_X b $.}
The function we created above simply stretches the beginning of the subinterval $ (c, d) $, up to $ a $ up to $ b $, and compresses the latter part of the subinterval, from $ a $ to $ d $, down to $ b $, and of course sends $ a $ to $ b $, leaving the rest of the real line that's not in the subinterval untouched. Let's now move on to analyzing the definable sets of $ (\R, <) $, which are also connected to intervals. Further, the equivalence relation $ \sim_X $ will also be useful as it allows us to consider automorphisms that fix $ X $, which are precisely the mappings which preserve definable sets.
\nn
Given that our language is a singleton with the proposition
    $$ P, \ti{ where } \pi(P) = 2, \st I(P) = \{ (a, b) \in \R^2 \st a < b \} $$
where $ I $ is our interpretation function, we are restricted to defining sets using the order relation. Then, any formula we construct will be some logical connectives over a combination ordering constraints, and so if we use the formula to define a set, the elements of the set satisfy some combination of ordering constraints. Elements satisfying ordering constraints are either single points or intervals of points with endpoints defined by the constraints, and so we'd expect sets defined this way to be collections of intervals (some of which may be infinite).
\nThm{Definable sets in $ (R, <) $ are built out of intervals}{Let $ X \subseteq $ be finite and $ A \subseteq \R $. Then $ A $ is definable in $ (\R, <) $ with parameters from $ X $ if and only if $ A $ is a finite union of intervals whose endpoints are in $ X $.}
\Pf{Clearly, if $ A $ is a finite union of intervals with endpoints in $ X $, we can construct the formula
    $$ \phi(x, x_1, \cdots, x_n) = \left( P(x, x_1) \lor \bigvee_{i = 1}^{n - 1} (P(x_i, x) \land P(x, x_{i + 1})) \lor P(x_n, x) \right) $$
so that for $ t \in \R $, $ \phi[t, t_1, \cdots, t_n] $, where $ X = \{ t_1, \cdots, t_n \} $ and $ t_i < t_{i + 1} $, states that $ t $ is in one of the intervals defined by the points of $ X $. For the other direction, if $ A $ is definable in $ (\R, <) $ with parameters from $ X $, then we can write
    $$ A = \{ t \in \R \st \phi[t, t_1, \cdots, t_n] \} $$
for some formula $ \phi $. If $ t \in A $, then it follows that every real number equivalent to $ t $ under $ \sim_X $ is in $ A $, simply because of how we defined $ \sim_X $ - it follows that if $ t \sim_X s $ then there's an automorphism $ e $ which fixes $ X $ and $ e(t) = s $. Since automorphisms that leave parameters invariant must leave definable sets invariant, $ t \in A \rightarrow e(t) = s \in A $, and so $ [t]_X \subseteq A $. This proves the theorem, since it follows that we can write
    $$ A = \bigcup_t \hspace{0.15cm} [t]_X $$
for some collection of real numbers $ t $, with each $ [t]_X $ being an interval (with endpoints in $ X $) as we proved before. Since $ X $ is finite, the equivalence classes of $ \sim_X $ partition the real line into finitely many intervals, and so $ A $ is the union of finitely many intervals whose endpoints are in $ X $ as desired.}
As a pleasant consequence, we can easily characterize the elementary substructures of $ (\R, <) $ by applying the Tarski-Vaught test to the above, concluding that $ \M \preceq (\R, <) $ (with the universe $ \M $ being a subset of the reals) if and only if $ (M, <_M) $ is a dense total order without endpoints (recall that a dense order $ < $ is an order on a set for which the order is arbitrarily finely grained, with no gaps; formally $ x < y \rightarrow \exists z \st x < z < y $). Of course, being a dense total order follows directly from the (topological) completeness of the reals and hence, being a first-order property of the real line, is a requisite for any elementary substructure. For the other direction, any definable set with parameters from $ M $ is the finite union of intervals whose endpoints are in $ M $; since $ \M $ is densely ordered, each of these intervals contains another point $ m \in M $, and so $ m \in A $; the condition of no endpoints is the account for unbounded intervals. We can use this corollary to prove the same theorem, that definable sets are built out of intervals (with endpoints in the parameterizing set), for $ (\Q, <) $ as well.

\section{Countable Sets}
Let's begin with some preliminary set theoretic definitions.
\nDefn{Countable}{A set $ A $ is \tb{countable} if there exists a surjective map from $ \N $ to $ A $.}
\Thm{Countability is closed under countable union}{If $ I $ is a countable index set, then and $ A_i $ is countable for $ i \in I $, then}
    $$ \bigcup_{i \in I} A_i $$
\InThm{is also countable.}
\Pf{TODO}
\Thm{Uncountabilty of the reals}{The set of real numbers are not countable.}
\Pf{TODO Cantor's diagonalization.}


\section{Lowenheim-Skolem Theorem}

\section{Arbitrary Dense Total Orders}

\end{document}

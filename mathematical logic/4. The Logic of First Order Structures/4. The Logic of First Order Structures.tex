\documentclass{article}

\usepackage[margin=0.5in]{geometry}
\usepackage{amsmath, amssymb, changepage, amsthm}

\begin{document}

\newcommand{\N}{\mathbb{N}}
\newcommand{\Z}{\mathbb{Z}}
\newcommand{\Q}{\mathbb{Q}}
\newcommand{\R}{\mathbb{R}}
\newcommand{\T}{\text{\normalfont\ T}}
\newcommand{\F}{\text{\normalfont\ F}}
\newcommand{\ti}{\textit}
\newcommand{\tb}{\textbf}
\newcommand{\n}{\leavevmode \newline}
\newcommand{\nn}{\leavevmode \newline \newline}
\def \Def#1#2{\begin{adjustwidth}{0.85cm}{0.85cm} \tb{(Definition) #1}: \ti{#2} \end{adjustwidth}}
\def \nDef#1#2{\n \Def{#1}{#2}}
\def \Defn#1#2{\Def{#1}{#2} \n}
\def \nDefn#1#2{\n \Defn{#1}{#2}}
\def \Defcont#1{\begin{adjustwidth}{0.85cm}{0.85cm} \ti{#1} \end{adjustwidth} \n}
\def \InDef#1{\ti{\begin{adjustwidth}{0.85cm}{0.85cm} #1 \end{adjustwidth}}}
\def \Thm#1#2{\begin{adjustwidth}{0.85cm}{0.85cm} \tb{(Theorem) #1}: \ti{#2} \end{adjustwidth}}
\def \nThm#1#2{\n \Thm{#1}{#2}}
\def \Thmn#1#2{\Thm{#1}{#2} \n}
\def \nThmn#1#2{\n \Thmn{#1}{#2}}
\def \InThm#1{\ti{\begin{adjustwidth}{0.85cm}{0.85cm} #1 \end{adjustwidth}}}
\def \Pf#1{\begin{adjustwidth}{0.85cm}{0.85cm} \textit{Proof}: #1 \qedsymbol \end{adjustwidth} \n}
\newcommand{\st}{\textnormal{ s.t. }}
\newcommand{\proplang}{\mathcal{L}_0}
\newcommand{\predlang}{\mathcal{L}}
\newcommand{\M}{\mathcal{M}}
\newcommand{\A}{\mathcal{A}}
\newcommand{\LA}{\predlang_\A}

\title{Logic of First Order Structures}
\author{Piyush Patil}
\date{October 9, 2017}
\maketitle

\section{Isomorphisms between structures}
We begin by defining isomorphisms, i.e. structure-preserving bijective mappings, between structures.
\nDef{$ \LA $-structure isomorphism}{Let $ \M = (M, I) $ and $ \mathcal{N} = (N, J) $ be $ \LA $-structures. An \tb{isomorphism} between $ \M $ and $ \mathcal{N} $ is a bijective mapping $ f: M \rightarrow N $ for which the following hold:}
\InDef{\begin{enumerate}
    \item Preservation of constant symbols: For each constant $ c_i $ in the domain of $ I $,
        $$ f(I(c_i)) = J(c_i) $$
    \item Preservation of function symbols: For each function symbol $ F_i $ in the domain of $ I $,
        $$ \ti{ for every } a_1, \cdots, a_{\pi(F_i)}, b \in M: I(F_i)(a_1, \cdots, a_{\pi(F_i)}) = b \ti{ if and only if } J(F_i)(f(a_1), \cdots, f(a_{\pi(F_i)})) = b $$
    \item Preservation of predicate symbols: For each preservation symbol $ P_i $ in the domain of $ I $,
        $$ \ti{ for every } a_1, \cdots, a_{\pi(P_i)} \in M: (a_1, \cdots, a_{\pi(P_i)}) \in I(P_i) \ti{ if and only if } (f(a_1), \cdots, f(a_{\pi(P_i)})) \in J(P_i) $$
\end{enumerate}}
As usual, a morphism is simply a structure-preserving mapping between general mathematical objects (or more precisely, categories), and an isomorphism is a sufficiently strong such mapping for which we can conclude that the two objects are in fact the same, structurally. Given two structures for which we can associate every object in the universe of one with a unique object in the universe of the other, and moreover for which the interpretation of constant, function, and predicate symbols is the same with respect to the respective objects, it follows that the structures are merely relabelings of each other. Note that isomorphisms are closed under composition and inverses.
\nn
To emphasize that isomorphisms are structure-preserving, it should be true that they preserve satisfiability. We prove this in the following theorem, which tells us that if we replace a universe's objects with the associated (through an isomorphism) objects of a different universes, satisfiability is preserved under the image of the isomorphism. In general, remapping objects like this might destroy the structure and render certain formulas satisfied or unsatisfied differently from the source universe, but because isomorphisms are designed to preserve logical structure, this is not the case.
\nThm{Isomorphisms preserve satisfiability}{Suppose $ f: M \rightarrow N $ is an isomorphism between the $ \LA $-structures $ \M = (M, I) $ and $ \mathcal{N} = (N, J) $. If $ \nu $ is any $ \M $-assignment, then the composition $ f \circ \nu $ is an $ \mathcal{N} $-assignment, and further}
$$ \ti{ for every $ \LA $-formula $ \phi $: } (\M, \nu) \vDash \phi \ti{ if and only if } (\mathcal{N}, f \circ \nu) \vDash \phi $$
\Pf{It's clear that $ f \circ \nu $ is an $ \mathcal{N} $-assignment, since it first uses $ \nu $ to map variables to objects in $ M $, which are then uniquely mapped to objects in $ N $ by $ f $. Our outline for proving that the composition preserves satisfiability is to first prove, by induction, that the composition of an assignment with an isomorphism preserves the interpretation of terms, and then use that result to prove the theorem itself, by induction.
\n
Recall that given our assignment $ \nu $, which assigns variables to objects in $ M $, the extended assignment leverages our variable assignments to assign more general terms to objects in $ M $. We wish to prove that $ f \circ \overline{\nu} = \overline{f \circ \nu} $. Let's proceed by induction on the length of a term. The base case, then, is analogous to the case of atomic terms, in which case
    $$ (f \circ \overline{\nu})(\langle c_i \rangle) = f(\overline{\nu}(\langle c_i \rangle)) = f(I(c_i)) = J(c_i) = \overline{f \circ \nu}(\langle c_i \rangle) $$
since all extended assignments over $ N $ by definiton map constants to $ J(c_i) $, or
    $$ (f \circ \overline{\nu})(\langle x_i \rangle) = f(\overline{\nu}(\langle x_i \rangle)) = f(\nu(x_i)) = (f \circ \nu)(x_i) = \overline{f \circ \nu}(x_i) $$
Next, suppose that indeed the interpretation of terms if preserved for terms of length less than that of a fixed term $ \tau $. Since $ \tau $ is not atomic by assumption, we can write $ \tau = F_i(\tau_1, \cdots, \tau_n) $ for terms $ \tau_j, 1 \leq j \leq n $, shorter than $ \tau $. Then
    $$ \begin{aligned}
        (f \circ \overline{\nu})(\tau) &= f(\overline{\nu}(F_i(\tau_1, \cdots, \tau_n))) = f(I(F_i)(\overline{\nu}(\tau_1), \cdots, \overline{\nu}(\tau_n))) = f(J(F_i)(f(\overline{\nu}(\tau_1)), \cdots, f(\overline{\nu}(\tau_n)))) \\
        &= f(J(F_i)(\overline{f \circ \nu}(\tau_1), \cdots, \overline{f \circ \nu}(\tau_n))) = \overline{f \circ \nu}(F_i(\tau_1, \cdots, \tau_n))
    \end{aligned} $$
This proves that $ \overline{f \circ \nu} = f \circ \overline{\nu} $.
\n
Next, let's use this result to prove the analogous result on formulas, again on induction on the length of formulas. Again, the base case corresponds to atomic formulas, which easily follow: TODO}
The above theorem suggests that isomorphic structures are equivalent, at least in terms of which formulas they satisfy and which they don't. More trivially, if it's the case that two structures satisfy precisely the exact same sentences (a special case of the above theorem that's independent of assignments), we say the structures, say $ \M $ and $ \mathcal{M} $, are \ti{elementarily equivalent}, and write $ \M \equiv \mathcal{N} $.

\section{Substructures and elementary substructures}

\section{Definable Sets and Tarski's Criterion}

\section{Dense Orders}

\section{Countable Sets}

\section{Lowenheim-Skolem Theorem}

\section{Arbitrary Dense Total Orders}

\end{document}

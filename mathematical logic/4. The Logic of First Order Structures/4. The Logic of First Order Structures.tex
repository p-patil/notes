\documentclass{article}

\usepackage[margin=0.5in]{geometry}
\usepackage{amsmath, amssymb, changepage, amsthm}

\begin{document}

\newcommand{\N}{\mathbb{N}}
\newcommand{\Z}{\mathbb{Z}}
\newcommand{\Q}{\mathbb{Q}}
\newcommand{\R}{\mathbb{R}}
\newcommand{\T}{\text{\normalfont\ T}}
\newcommand{\F}{\text{\normalfont\ F}}
\newcommand{\ti}{\textit}
\newcommand{\tb}{\textbf}
\newcommand{\n}{\leavevmode \newline}
\newcommand{\nn}{\leavevmode \newline \newline}
\def \Def#1#2{\begin{adjustwidth}{0.85cm}{0.85cm} \tb{(Definition) #1}: \ti{#2} \end{adjustwidth}}
\def \nDef#1#2{\n \Def{#1}{#2}}
\def \Defn#1#2{\Def{#1}{#2} \n}
\def \nDefn#1#2{\n \Defn{#1}{#2}}
\def \Defcont#1{\begin{adjustwidth}{0.85cm}{0.85cm} \ti{#1} \end{adjustwidth} \n}
\def \InDef#1{\ti{\begin{adjustwidth}{0.85cm}{0.85cm} #1 \end{adjustwidth}}}
\def \Thm#1#2{\begin{adjustwidth}{0.85cm}{0.85cm} \tb{(Theorem) #1}: \ti{#2} \end{adjustwidth}}
\def \nThm#1#2{\n \Thm{#1}{#2}}
\def \Thmn#1#2{\Thm{#1}{#2} \n}
\def \nThmn#1#2{\n \Thmn{#1}{#2}}
\def \InThm#1{\ti{\begin{adjustwidth}{0.85cm}{0.85cm} #1 \end{adjustwidth}}}
\def \Pf#1{\begin{adjustwidth}{0.85cm}{0.85cm} \textit{Proof}: #1 \qedsymbol \end{adjustwidth} \n}
\newcommand{\st}{\textnormal{ s.t. }}
\newcommand{\proplang}{\mathcal{L}_0}
\newcommand{\predlang}{\mathcal{L}}
\newcommand{\M}{\mathcal{M}}
\newcommand{\A}{\mathcal{A}}
\newcommand{\LA}{\predlang_\A}

\title{Logic of First Order Structures}
\author{Piyush Patil}
\date{October 9, 2017}
\maketitle

\section{Isomorphisms between structures}
We begin by defining isomorphisms, i.e. structure-preserving bijective mappings, between structures.
\nDef{$ \LA $-structure isomorphism}{Let $ \M = (M, I) $ and $ \mathcal{N} = (N, J) $ be $ \LA $-structures. An \tb{isomorphism} between $ \M $ and $ \mathcal{N} $ is a bijective mapping $ f: M \rightarrow N $ for which the following hold:}
\InDef{\begin{enumerate}
    \item Preservation of constant symbols: For each constant $ c_i $ in the domain of $ I $,
        $$ f(I(c_i)) = J(c_i) $$
    \item Preservation of function symbols: For each function symbol $ F_i $ in the domain of $ I $,
        $$ \ti{ for every } a_1, \cdots, a_{\pi(F_i)}, b \in M: I(F_i)(a_1, \cdots, a_{\pi(F_i)}) = b \ti{ if and only if } J(F_i)(f(a_1), \cdots, f(a_{\pi(F_i)})) = f(b) $$
    \item Preservation of predicate symbols: For each preservation symbol $ P_i $ in the domain of $ I $,
        $$ \ti{ for every } a_1, \cdots, a_{\pi(P_i)} \in M: (a_1, \cdots, a_{\pi(P_i)}) \in I(P_i) \ti{ if and only if } (f(a_1), \cdots, f(a_{\pi(P_i)})) \in J(P_i) $$
\end{enumerate}}
We often write $ \M \cong \mathcal{N} $ to denote that $ \M $ is isomorphic to $ \mathcal{N} $. As usual, a morphism is simply a structure-preserving mapping between general mathematical objects (or more precisely, categories), and an isomorphism is a sufficiently strong such mapping for which we can conclude that the two objects are in fact the same, structurally. Given two structures for which we can associate every object in the universe of one with a unique object in the universe of the other, and moreover for which the interpretation of constant, function, and predicate symbols is the same with respect to the respective objects, it follows that the structures are merely relabelings of each other. Note that isomorphisms are closed under composition and inverses.
\nn
To emphasize that isomorphisms are structure-preserving, it should be true that they preserve satisfiability. We prove this in the following theorem, which tells us that if we replace a universe's objects with the associated (through an isomorphism) objects of a different universes, satisfiability is preserved under the image of the isomorphism. In general, remapping objects like this might destroy the structure and render certain formulas satisfied or unsatisfied differently from the source universe, but because isomorphisms are designed to preserve logical structure, this is not the case.
\nThm{Isomorphisms preserve satisfiability}{Suppose $ f: M \rightarrow N $ is an isomorphism between the $ \LA $-structures $ \M = (M, I) $ and $ \mathcal{N} = (N, J) $. If $ \nu $ is any $ \M $-assignment, then the composition $ f \circ \nu $ is an $ \mathcal{N} $-assignment, and further}
$$ \ti{ for every $ \LA $-formula $ \phi $: } (\M, \nu) \vDash \phi \ti{ if and only if } (\mathcal{N}, f \circ \nu) \vDash \phi $$
\Pf{It's clear that $ f \circ \nu $ is an $ \mathcal{N} $-assignment, since it first uses $ \nu $ to map variables to objects in $ M $, which are then uniquely mapped to objects in $ N $ by $ f $. Our outline for proving that the composition preserves satisfiability is to first prove, by induction, that the composition of an assignment with an isomorphism preserves the interpretation of terms, and then use that result to prove the theorem itself, by induction.
\n
Recall that given our assignment $ \nu $, which assigns variables to objects in $ M $, the extended assignment leverages our variable assignments to assign more general terms to objects in $ M $. We wish to prove that $ f \circ \overline{\nu} = \overline{f \circ \nu} $. Let's proceed by induction on the length of a term. The base case, then, is analogous to the case of atomic terms, in which case
    $$ (f \circ \overline{\nu})(\langle c_i \rangle) = f(\overline{\nu}(\langle c_i \rangle)) = f(I(c_i)) = J(c_i) = \overline{f \circ \nu}(\langle c_i \rangle) $$
since all extended assignments over $ N $ by definiton map constants to $ J(c_i) $, or
    $$ (f \circ \overline{\nu})(\langle x_i \rangle) = f(\overline{\nu}(\langle x_i \rangle)) = f(\nu(x_i)) = (f \circ \nu)(x_i) = \overline{f \circ \nu}(x_i) $$
Next, suppose that indeed the interpretation of terms if preserved for terms of length less than that of a fixed term $ \tau $. Since $ \tau $ is not atomic by assumption, we can write $ \tau = F_i(\tau_1, \cdots, \tau_n) $ for terms $ \tau_j, 1 \leq j \leq n $, shorter than $ \tau $. Then
    $$ \begin{aligned}
        (f \circ \overline{\nu})(\tau) &= f(\overline{\nu}(F_i(\tau_1, \cdots, \tau_n))) = f(I(F_i)(\overline{\nu}(\tau_1), \cdots, \overline{\nu}(\tau_n))) = f(J(F_i)(f(\overline{\nu}(\tau_1)), \cdots, f(\overline{\nu}(\tau_n)))) \\
        &= f(J(F_i)(\overline{f \circ \nu}(\tau_1), \cdots, \overline{f \circ \nu}(\tau_n))) = \overline{f \circ \nu}(F_i(\tau_1, \cdots, \tau_n))
    \end{aligned} $$
This proves that $ \overline{f \circ \nu} = f \circ \overline{\nu} $.
\n
Next, let's use this result to prove the analogous result on formulas, again on induction on the length of formulas. Again, the base case corresponds to atomic formulas, which easily follow: TODO}
The above theorem suggests that isomorphic structures are equivalent, at least in terms of which formulas they satisfy and which they don't. More trivially, if it's the case that two structures satisfy precisely the exact same sentences (a special case of the above theorem that's independent of assignments), we say the structures, say $ \M $ and $ \mathcal{M} $, are \ti{elementarily equivalent}, and write $ \M \equiv \mathcal{N} $.

\section{Substructures and elementary substructures}
In this section, we consider structures that contain other structures, and characterize different notions of containment and ways of determining substructure relationships.
\nDef{Substructure}{Let $ \M = (M, I) $ and $ \mathcal{N} = (N, J) $ be $ \LA $-structures. $ \M $ is a \tb{substructure} of $ \mathcal{N} $, denoted $ \M \subseteq \mathcal{N} $, if $ M \subseteq N $ and the following hold:}
\InDef{\begin{enumerate}
    \item Interpretations agree on constants: $ \forall c_i \in \mathcal{A}: I(c_i) = J(c_i) $.
    \item Interpretations agree on functions: $ \forall F_i \in \mathcal{A}: I(F_i) \text{ is the restriction of } J(F_i) \text{ to } M $.
    \item Interpretations agree on predicates: $ \forall P_i \in \mathcal{A}: I(P_i) = J(P_i) \cap M^{\pi(P_i)} $.
\end{enumerate}}
\n
It turns out that substructure relationships hold precisely when we have a form of elementary equivalence, which is strong enough to automatically guarantee the above properties.
\nThm{Substructures prove the same atomic formulas}{Let $ \M = (M, I) $ and $ \mathcal{N} = (N, J) $ be $ \LA $-structures with $ M \subseteq N $. Then $ \M \subseteq \mathcal{N} $ if and only if}
\InThm{
    $$ \forall \text{ atomic $ \LA $-formulas $ \phi $, $ \M $-assignments $ \nu $}: (\M, \nu) \vDash \phi \iff (\mathcal{N}, \nu) \vDash \phi $$
}
\Pf{TODO}
This theorem inspires the following definition, which formalizes the above property as a similar but distinct property from elementary equivalence.
\nDef{Elementary substructure}{Let $ \M $ and $ \mathcal{N} $ be $ \LA $-structures with $ \M \subseteq \mathcal{N} $. $ \M $ is an \tb{elementary substructure} of $ \mathcal{N} $ if}
\InDef{
    $$ \forall \ti{$ \LA $-formulas $ \phi $, $ \M $-assignments $ \nu $}: (\M, \nu) \vDash \phi \iff (\mathcal{N}, \nu) \vDash \phi $$
In which case we write $ \M \preceq \mathcal{N} $.}

\section{Definable Sets and Tarski's Criterion}
Let's start this section by introducing definable sets. We'll give the formal definition below, followed by an explanation of the motivation.
\nDef{Definable set}{Let $ \M = (M, I) $ be an $ \LA $-structure. We say that a set $ Y \subseteq M^n $ is \tb{definable in $ \M $ with parameters from $ X $} if there exists a set $ X \subseteq M $ with elements $ b_1, \cdots, b_m \in X $, and an $ \LA $-formula $ \phi $, such that}
\InDef{\begin{enumerate}
    \item $ \phi $ has $ n + m $ free variables $ x_1, \cdots, x_{n + m} $
    \item For every $ (a_1, \cdots, a_n) \in M^n $, $ (a_1, \cdots, a_n) \in Y $ if and only if there's some $ \M $-assignment $ \nu $ such that
        $$ \nu(x_i) = \begin{cases}
            a_i, &\ti{ if } 1 \leq i \leq n \\
            b_i, &\ti{ if } n + 1 \leq i \leq m
        \end{cases} $$
    and $ (\M, \nu) \vDash \phi $.
\end{enumerate}
Further, we say $ Y $ is \tb{definable in $ \M $ without parameters} if $ X = \emptyset $.}
Let's break this rather unwieldy definition down. Our motivation for what a definable set actually is comes from the following approach to naive set theory: informally, we conceptualize sets as collections of objects. Sets can be represented by explicitly enumerating their objects, but this approach has set theoretic shortcomings when we try to consider infinite, especially uncountable or larger, sets. Instead, sets are more often defined in terms of some kind of construction based on some property. Most generally, a set is defined as the collection of objects within some universe which satisfy some property. Symbolically, we may express this idea as
    $$ \ti{ for any set $ S $}: S = \{ x \st P(x) \} $$
for some predicate relation or proposition $ P $. Hence, set theory and logical systems are intimately related, as we require the former to lay the scaffolding of a truth system which allows us to formally construct predicates and propositions that we can use to define sets through some kind of filtering process over elements from some larger universe of elements (which itself is usually constructed axiomatically or taken as an undefined primitive). It may be initially unclear how this definition links to the motivation we just presented, and in particular how the component of definability with parameters fits in. The case of definability without parameters is simpler and easier to motivate, so we'll start with that before moving on to showing how the introduction of parameters is merely a generalization technique that enhances the expressivity of our logical system to define sets which are normally unable to be defined in the structure.
\nn
Recall that we initially motivated definability for sets from a set theoretic approach, in trying to answer the question of how to formally construct arbitrary, often infinite sets. To do so, we require some framework for a logical system that allows us to formally construct relations or some kind of logical filter for specifying which elements fit within our set and which do not. Within the context of first-order logic, $ \LA $-structures can be used in this regard. If have some $ \LA $-structure $ \M = (M, I) $, which we wish to use to define sets, then the total universe of objects from which our sets are constructed is $ M $, with $ I $ specifying the semantics of the underlying logical system. So, our question now becomes which subsets of $ M $ are definable and which aren't. Since we want to define sets as the collection of objects from the universe $ M $ which together share some property, we'll use predicates for this purpose. In fact, we can identify predicates $ P_i \in \A $ with sets in and of themselves, since sets are wholly defined by the membership operation, which we can leverage a predicate symbol for by taking it's truth value as an indication of the membership criterion. Hence, we want, informally, sets $ S \subseteq M $ to be defined using some notion similar to
    $$ S = \{ m \in M \st m \in I(P_i(m)) \ti{ for some $ i $ } \} $$
Of course, it seems unnecessarily restrictive to be forced to define sets using only predicates; there's no reason why we can't chain predicates or other logical symbols together to form more complex propositions that serve as the filtering mechanism in defining $ S $. For this reason, the definition we really want is something of the form
    $$ S = \{ m \in M \st \M \vDash \phi[m] \ti{ for some $ \LA$-formula $ \phi(x_1) $ } \} $$
In other words, we define $ S $ as the objects in $ M $ which make $ \phi $ true, under the logical structure defined by $ \M $. Notice that the definition that $ S \subseteq M $ is definable if there exists a mapping $ \nu $ and $ \LA $-formula $ \phi $ with one free variable $ x_1 $ such that $ m \in M $ is in $ S $, where
    $$ \nu(x_1) = m $$
if and only if $ (\M, \nu) \vDash \phi $ is equivalent to the formal definition we began this section with, in the case $ n = 1 $ and $ X = \emptyset $. We easily note we can leverage this definition to a more powerful definition that can capture definability for even more sets, not just the subsets of $ M $, if we consider subsets of $ M^n $. Extending the definition from sets of elements of $ M $ to sets of $ n $-tuples from $ M $ is straightforward; we just allow $ \phi $ to have $ n $ free variables and force $ \nu $ to assign each to the corresponding object in the tuple we're considering. This amounts to the definition
    $$ S \subseteq M^n \ti{ is definable if } \exists \phi(x_1, \cdots, x_n) \in \LA \st \forall (m_1, \cdots, m_n) \in M^n: (m_1, \cdots, m_n) \in S \iff \mathcal{M} \vDash \phi[m_1, \cdots, m_n] $$
One implication of this form of defining definability is that the same set might be definable under certain structures and languages, but not under others; the notion of definability depends critically on the underlying structure we use. More specifically, whether or not a set is definable depends on the exxpressivity of the underlying structure. A natural question to ask at this point is: if a set is not definable under some structure, is there some way we can extend the structure to a more expressive one under which our set can be defined. This leads us to the idea of the expansion of a structure. If we have some structure $ \M $ and some subset $ A $ of $ M $, we can augment the langauge $ \A $ as
    $$ \LA := \LA \cup \{ c_a, a \in A \} $$
    $$ I(c_a) := a $$
In other words, for each object in $ A $, we explicitly add a constant symbol to our language which we associate with the corresponding object in the augmented $ M $. This new structure, which we might denote $ \M_a $, is called the \ti{expansion} of $ \M $, and our reasoning for doing this is to enhance the expressivity of our logical structure. In particular, we can now define every finite subset of $ A $ by virtue of the new constant symbols we added to our language. Further, because we only ever use finitely many symbols when defining a set, any definable set in $ \M_a $ is also definable in $ \M_{A_0} $ for some subset $ A_0 \subseteq A $. This is exactly what we mean by definability with parameters - even if a set $ S \subseteq M $ is not definable in the raw $ \M $, if we can expand $ \M $ to some $ \M_A $ then $ S $ is definable in $ \M $ if we only allow it to use parameters from $ A $, given by the finite $ A_0 \subseteq A $. Thus the "parameters" in use are simply new constant symbols we add to our language which allow us to construct a previously impossible formula that uniquely decides which elements are in $ S $. That is, even though we can't find any $ \phi(x_1) $ for which $ m \in S \iff \phi[m] $ is true in $ \M $, we might be able to generalize $ \phi $ to $ \phi(x_1, x_2, \cdots, x_{m + 1}) $ and subsequently define $ S $ using $ m \in S \iff \phi[m, a_1, \cdots, a_m] $ is true $ \M_{A_0} $ where $ A_0 = \{ a_1, \cdots, a_n \} $.
\nn
We'll often make use of the equivalent, but more elegant definition of definability that $ Y \subseteq M^n $ is definable in $ \M = (M, I) $ with parameters from $ X \subseteq M $ if
    $$ \exists b_1, \cdots, b_m \in X, \phi(x_1, \cdots, x_{n + m}) \in \LA \st Y = \big \{ (a_1, \cdots, a_n) \in M^n \st \M \vDash \phi[a_1, \cdots, a_n, b_1, \cdots, b_m] \big \} $$
If a set is defined with parameters, then the set's definition depends only on and is determined completely by those parameters. Thus, it follows that any rearrangement of the objects of our universe should preserve both the definability of a set and the actual members of the set, so long as the rearrangement (1) preserves the logical structure of the underlying universe and interpretation, and (2) leaves the parameters defining the set unchanged. These two conditions are sufficient to determine the preservation of a definable set across any rearrangement of the universe, since the definability of the set, and therefore the set itself, hinges only on those parameters and therefore leaving the parameters unchanged necessitates leaving the set invariant through the rearrangement. Formally, we have the following theorem.
\nThm{Automorphisms that preserve parameters preserve definable sets}{Let $ \M = (M, I) $ be an $ \LA $-structure. If $ Y \subseteq M^n $ is definable in $ \M $ with parameters from $ X \subseteq M $, then for any automorphism $ f $ on $ \M $ which is invariant on $ X $, $ Y $ is preserved under $ f $. That is,}
    $$ \ti{ $ \forall $ automorphisms $ f: M \rightarrow M $}: \forall b \in X: f(b) = b \rightarrow  Y = f(Y) := \{ (f(a_1), \cdots, f(a_n)) \ti{ for } (a_1, \cdots, a_n) \in Y \} $$
\Pf{By definition, we can find $ b_1, \cdots, b_m \in X $ and $ \phi(x_1, \cdots, x_{n + m}) $ such that
    $$ (a_1, \cdots, a_n) \in Y \iff \M \vDash \phi(a_1, \cdots, a_n, b_1, \cdots, b_m) $$
Let's first prove that $ Y \subseteq f(Y) $. For any $ (a_1, \cdots, a_n) \in Y $, since $ \phi(a_1, \cdots, a_n, b_1, \cdots, b_m) $ is true in $ \M $, we can unpack the definition of satisfiability so that for any $ \M $-assignment $ \nu $ for which $ \nu(x_i) = a_i, 1 \leq i \leq n $ and $ \nu(x_j) = b_j, 1 \leq j \leq m $, $ (\M, \nu) \vDash \phi $; since isomorphisms preserve satisfiability, in the sense that for any isomorphism $ f $ to structure $ \mathcal{N} $, $ (\mathcal{N}, f \circ \nu) \vDash \phi $. Thus, if $ f $ is an automorphism then it follows that
    $$ \M \vDash \phi[f(a_1), \cdots, f(a_n), f(b_1), \cdots, f(b_m)] = \phi[f(a_1), \cdots, f(a_n), b_1, \cdots, b_m] $$
where the last equality comes from $ X $ being invariant under $ f $. Thus, $ (f(a_1), \cdots, f(a_n)) \in Y $, and since every element of $ f(Y) $ can be expressed in this form, it follows that $ f(Y) \subseteq Y $.
\nn
Let's now prove that $ Y \subseteq f(Y) $. Given some $ (a_1, \cdots, a_n) \in Y $, we want to show that $ a_i = f(a_i') $ for some $ (a_1', \cdots, a_n') \in Y $. Since $ f $ is an automorphism, so is the inverse mapping $ f^{-1} $. This means that since $ \M \vDash \phi[a_1, \cdots, a_n, b_1, \cdots, b_m] $, we also have that $ \M \vDash \phi[f^{-1}(a_1), \cdots, f^{-1}(a_n), b_1, \cdots, b_m] $ by the same logic as before. Hence, $ (f^{-1}(a_1), \cdots, f^{-1}(a_n)) \in Y $, and so we can take $ a_i' := f^{-1}(a_i) $ and conclude that $ Y \subseteq f(Y) $. It follows that $ Y = f(Y) $ as desired.
}


\section{Dense Orders}

\section{Countable Sets}

\section{Lowenheim-Skolem Theorem}

\section{Arbitrary Dense Total Orders}

\end{document}
